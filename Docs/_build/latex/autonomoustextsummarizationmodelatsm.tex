%% Generated by Sphinx.
\def\sphinxdocclass{report}
\documentclass[letterpaper,10pt,english]{sphinxmanual}
\ifdefined\pdfpxdimen
   \let\sphinxpxdimen\pdfpxdimen\else\newdimen\sphinxpxdimen
\fi \sphinxpxdimen=.75bp\relax
\ifdefined\pdfimageresolution
    \pdfimageresolution= \numexpr \dimexpr1in\relax/\sphinxpxdimen\relax
\fi
%% let collapsible pdf bookmarks panel have high depth per default
\PassOptionsToPackage{bookmarksdepth=5}{hyperref}

\PassOptionsToPackage{warn}{textcomp}
\usepackage[utf8]{inputenc}
\ifdefined\DeclareUnicodeCharacter
% support both utf8 and utf8x syntaxes
  \ifdefined\DeclareUnicodeCharacterAsOptional
    \def\sphinxDUC#1{\DeclareUnicodeCharacter{"#1}}
  \else
    \let\sphinxDUC\DeclareUnicodeCharacter
  \fi
  \sphinxDUC{00A0}{\nobreakspace}
  \sphinxDUC{2500}{\sphinxunichar{2500}}
  \sphinxDUC{2502}{\sphinxunichar{2502}}
  \sphinxDUC{2514}{\sphinxunichar{2514}}
  \sphinxDUC{251C}{\sphinxunichar{251C}}
  \sphinxDUC{2572}{\textbackslash}
\fi
\usepackage{cmap}
\usepackage[T1]{fontenc}
\usepackage{amsmath,amssymb,amstext}
\usepackage{babel}



\usepackage{tgtermes}
\usepackage{tgheros}
\renewcommand{\ttdefault}{txtt}



\usepackage[Bjarne]{fncychap}
\usepackage{sphinx}

\fvset{fontsize=auto}
\usepackage{geometry}


% Include hyperref last.
\usepackage{hyperref}
% Fix anchor placement for figures with captions.
\usepackage{hypcap}% it must be loaded after hyperref.
% Set up styles of URL: it should be placed after hyperref.
\urlstyle{same}

\addto\captionsenglish{\renewcommand{\contentsname}{Contents:}}

\usepackage{sphinxmessages}
\setcounter{tocdepth}{1}



\title{Autonomous Text Summarization Model (ATSM)}
\date{Jun 26, 2022}
\release{1.0}
\author{Roque Rodríguez, Adrian Lebreualt}
\newcommand{\sphinxlogo}{\vbox{}}
\renewcommand{\releasename}{Release}
\makeindex
\begin{document}

\ifdefined\shorthandoff
  \ifnum\catcode`\=\string=\active\shorthandoff{=}\fi
  \ifnum\catcode`\"=\active\shorthandoff{"}\fi
\fi

\pagestyle{empty}
\sphinxmaketitle
\pagestyle{plain}
\sphinxtableofcontents
\pagestyle{normal}
\phantomsection\label{\detokenize{index::doc}}


\sphinxstepscope


\chapter{Proyecto de grado}
\label{\detokenize{modules:proyecto-de-grado}}\label{\detokenize{modules::doc}}
\sphinxstepscope


\section{Summarizer module}
\label{\detokenize{Summarizer:module-Summarizer}}\label{\detokenize{Summarizer:summarizer-module}}\label{\detokenize{Summarizer::doc}}\index{module@\spxentry{module}!Summarizer@\spxentry{Summarizer}}\index{Summarizer@\spxentry{Summarizer}!module@\spxentry{module}}
\sphinxAtStartPar
Automatically generated by Colaboratory.
\begin{description}
\sphinxlineitem{Original file is located at}
\sphinxAtStartPar
\sphinxurl{https://colab.research.google.com/drive/1DHCRkfRBOPS1bEibUA-1vPdfDqvQWuJe}

\end{description}

\sphinxAtStartPar
Este proyecto requiere la instalación de los paquetes Transformers y Pytorch (Torch).
Para instalar los paquetes usar:

\sphinxAtStartPar
pip install transformers
pip install torch
\index{ATSM (class in Summarizer)@\spxentry{ATSM}\spxextra{class in Summarizer}}

\begin{fulllineitems}
\phantomsection\label{\detokenize{Summarizer:Summarizer.ATSM}}
\pysigstartsignatures
\pysigline{\sphinxbfcode{\sphinxupquote{class\DUrole{w}{  }}}\sphinxcode{\sphinxupquote{Summarizer.}}\sphinxbfcode{\sphinxupquote{ATSM}}}
\pysigstopsignatures
\sphinxAtStartPar
Bases: \sphinxcode{\sphinxupquote{object}}

\sphinxAtStartPar
Clase ATSM encargada de los metodos para la generación del resumen.
\index{Settings\_ENG() (Summarizer.ATSM method)@\spxentry{Settings\_ENG()}\spxextra{Summarizer.ATSM method}}

\begin{fulllineitems}
\phantomsection\label{\detokenize{Summarizer:Summarizer.ATSM.Settings_ENG}}
\pysigstartsignatures
\pysiglinewithargsret{\sphinxbfcode{\sphinxupquote{Settings\_ENG}}}{}{}
\pysigstopsignatures
\sphinxAtStartPar
Metodo para instancias las configuraciones del idioma inglés.
Utiliza el algoritmos de BERT CNN Summarization.
\begin{description}
\sphinxlineitem{Inputs:}
\sphinxAtStartPar
:N/A

\sphinxlineitem{Returns:}\begin{quote}\begin{description}
\sphinxlineitem{tokenizer}
\sphinxAtStartPar
Herramienta para tokenizar el texto.

\sphinxlineitem{model}
\sphinxAtStartPar
modelo de NLP

\sphinxlineitem{device}
\sphinxAtStartPar
configuraciones del dispositivo

\end{description}\end{quote}

\end{description}

\end{fulllineitems}

\index{Settings\_ESP() (Summarizer.ATSM method)@\spxentry{Settings\_ESP()}\spxextra{Summarizer.ATSM method}}

\begin{fulllineitems}
\phantomsection\label{\detokenize{Summarizer:Summarizer.ATSM.Settings_ESP}}
\pysigstartsignatures
\pysiglinewithargsret{\sphinxbfcode{\sphinxupquote{Settings\_ESP}}}{}{}
\pysigstopsignatures
\sphinxAtStartPar
Metodo para instancias las configuraciones del idioma español.
Utiliza el algoritmos de BERT Fined Tuned Summarization.
\begin{description}
\sphinxlineitem{Inputs:}
\sphinxAtStartPar
:N/A

\sphinxlineitem{Returns:}\begin{quote}\begin{description}
\sphinxlineitem{tokenizer}
\sphinxAtStartPar
Herramienta para tokenizar el texto.

\sphinxlineitem{model}
\sphinxAtStartPar
modelo de NLP

\sphinxlineitem{device}
\sphinxAtStartPar
configuraciones del dispositivo

\end{description}\end{quote}

\end{description}

\end{fulllineitems}

\index{chunker() (Summarizer.ATSM method)@\spxentry{chunker()}\spxextra{Summarizer.ATSM method}}

\begin{fulllineitems}
\phantomsection\label{\detokenize{Summarizer:Summarizer.ATSM.chunker}}
\pysigstartsignatures
\pysiglinewithargsret{\sphinxbfcode{\sphinxupquote{chunker}}}{\emph{\DUrole{n}{sentences}}}{}
\pysigstopsignatures
\sphinxAtStartPar
Metodo Fragmentador de textos. Toma textos largos y los convierte en pedazos mas asumibles para el modelo.
\begin{description}
\sphinxlineitem{Inputs:}\begin{quote}\begin{description}
\sphinxlineitem{max\_chunk}
\sphinxAtStartPar
Maximo numero de palabras por fragmento.

\sphinxlineitem{sentences}
\sphinxAtStartPar
Lista de oraciones.

\end{description}\end{quote}

\sphinxlineitem{Returns:}\begin{quote}\begin{description}
\sphinxlineitem{chunks}
\sphinxAtStartPar
Lista de fragmentos.

\end{description}\end{quote}

\end{description}

\end{fulllineitems}

\index{generate\_sentences() (Summarizer.ATSM method)@\spxentry{generate\_sentences()}\spxextra{Summarizer.ATSM method}}

\begin{fulllineitems}
\phantomsection\label{\detokenize{Summarizer:Summarizer.ATSM.generate_sentences}}
\pysigstartsignatures
\pysiglinewithargsret{\sphinxbfcode{\sphinxupquote{generate\_sentences}}}{}{}
\pysigstopsignatures
\sphinxAtStartPar
Metodo para dividir el texto por oraciones.
\begin{description}
\sphinxlineitem{Inputs:}\begin{quote}\begin{description}
\sphinxlineitem{text}
\sphinxAtStartPar
Texto original

\end{description}\end{quote}

\sphinxlineitem{Returns:}\begin{quote}\begin{description}
\sphinxlineitem{sentences}
\sphinxAtStartPar
Lista de oraciones.

\end{description}\end{quote}

\end{description}

\end{fulllineitems}

\index{generate\_summary() (Summarizer.ATSM method)@\spxentry{generate\_summary()}\spxextra{Summarizer.ATSM method}}

\begin{fulllineitems}
\phantomsection\label{\detokenize{Summarizer:Summarizer.ATSM.generate_summary}}
\pysigstartsignatures
\pysiglinewithargsret{\sphinxbfcode{\sphinxupquote{generate\_summary}}}{\emph{\DUrole{n}{tokenizer}}, \emph{\DUrole{n}{model}}, \emph{\DUrole{n}{device}}, \emph{\DUrole{n}{max\_length}}, \emph{\DUrole{n}{min\_length}}}{}
\pysigstopsignatures
\sphinxAtStartPar
Método para generar los resúmenes de los fragmentos.
\begin{description}
\sphinxlineitem{Inputs:}\begin{quote}\begin{description}
\sphinxlineitem{text}
\sphinxAtStartPar
Texto que se desea resumir.

\sphinxlineitem{tokenizer}
\sphinxAtStartPar
Tokenizador para el texto.

\sphinxlineitem{model}
\sphinxAtStartPar
Modelo NLP

\sphinxlineitem{device}
\sphinxAtStartPar
Condifguraciones de dispostivo.

\sphinxlineitem{max\_length}
\sphinxAtStartPar
Longitud máxima del resúmen.

\sphinxlineitem{min\_length}
\sphinxAtStartPar
Longitud mínima del resúmen.

\end{description}\end{quote}

\sphinxlineitem{Returns:}\begin{quote}\begin{description}
\sphinxlineitem{summary}
\sphinxAtStartPar
Texto resumido según los parámetros.

\end{description}\end{quote}

\end{description}

\end{fulllineitems}

\index{max\_tokens (Summarizer.ATSM attribute)@\spxentry{max\_tokens}\spxextra{Summarizer.ATSM attribute}}

\begin{fulllineitems}
\phantomsection\label{\detokenize{Summarizer:Summarizer.ATSM.max_tokens}}
\pysigstartsignatures
\pysigline{\sphinxbfcode{\sphinxupquote{max\_tokens}}\sphinxbfcode{\sphinxupquote{\DUrole{w}{  }\DUrole{p}{=}\DUrole{w}{  }6000}}}
\pysigstopsignatures
\end{fulllineitems}

\index{summarize() (Summarizer.ATSM method)@\spxentry{summarize()}\spxextra{Summarizer.ATSM method}}

\begin{fulllineitems}
\phantomsection\label{\detokenize{Summarizer:Summarizer.ATSM.summarize}}
\pysigstartsignatures
\pysiglinewithargsret{\sphinxbfcode{\sphinxupquote{summarize}}}{\emph{\DUrole{n}{text}}, \emph{\DUrole{n}{max\_length}}, \emph{\DUrole{n}{min\_length}}}{}
\pysigstopsignatures
\sphinxAtStartPar
Método que unifica todas las configuraciones, pre\sphinxhyphen{}procesamiento y la generación del resúmen.
\begin{description}
\sphinxlineitem{Inputs:}\begin{quote}\begin{description}
\sphinxlineitem{lang}
\sphinxAtStartPar
Idioma del texto.

\sphinxlineitem{text}
\sphinxAtStartPar
Texto que se desea resumir.

\sphinxlineitem{max\_length}
\sphinxAtStartPar
Longitud máxima del resumen.

\sphinxlineitem{min\_length}
\sphinxAtStartPar
Longitud mínima del resumen.

\end{description}\end{quote}

\sphinxlineitem{Returns:}\begin{quote}\begin{description}
\sphinxlineitem{summary}
\sphinxAtStartPar
Texto resumido según los parámetros.

\end{description}\end{quote}

\end{description}

\end{fulllineitems}


\end{fulllineitems}



\chapter{Indices and tables}
\label{\detokenize{index:indices-and-tables}}\begin{itemize}
\item {} 
\sphinxAtStartPar
\DUrole{xref,std,std-ref}{genindex}

\item {} 
\sphinxAtStartPar
\DUrole{xref,std,std-ref}{modindex}

\item {} 
\sphinxAtStartPar
\DUrole{xref,std,std-ref}{search}

\end{itemize}


\renewcommand{\indexname}{Python Module Index}
\begin{sphinxtheindex}
\let\bigletter\sphinxstyleindexlettergroup
\bigletter{s}
\item\relax\sphinxstyleindexentry{Summarizer}\sphinxstyleindexpageref{Summarizer:\detokenize{module-Summarizer}}
\end{sphinxtheindex}

\renewcommand{\indexname}{Index}
\printindex
\end{document}